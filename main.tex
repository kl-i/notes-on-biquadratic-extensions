\documentclass{article}
\usepackage[left=1in,right=1in]{geometry}
\usepackage{subfiles}
\usepackage{amsmath, amssymb, stmaryrd, verbatim} % math symbols
\usepackage{amsthm} % thm environment
\usepackage{mdframed} % Customizable Boxes
\usepackage{hyperref,nameref,cleveref,enumitem} % for references, hyperlinks
\usepackage[dvipsnames]{xcolor} % Fancy Colours
\usepackage{mathrsfs} % Fancy font
\usepackage{bbm} % mathbb numerals
\usepackage{tikz, tikz-cd, float} % Commutative Diagrams
\usetikzlibrary{decorations.pathmorphing} % for squiggly arrows in tikzcd
\usepackage{perpage}
\usepackage{parskip} % So that paragraphs look nice
\usepackage{ifthen,xargs} % For defining better commands
\usepackage[T1]{fontenc}
\usepackage[utf8]{inputenc}
\usepackage{tgpagella}
\usepackage{cancel}

% Bibliography
\usepackage{url}
\usepackage{biblatex}

\addbibresource{mybib.bib}

% Shortcuts

% % Local to this project
\renewcommand\labelitemi{--} % Makes itemize use dashes instead of bullets

% % Misc
\newcommand{\brkt}[1]{\left(#1\right)}
\newcommand{\sqbrkt}[1]{\left[#1\right]}
\newcommand{\dash}{\text{-}}

% % Logic
\renewcommand{\implies}{\Rightarrow}
\renewcommand{\iff}{\Leftrightarrow}
\newcommand{\limplies}{\Leftarrow}
\newcommand{\NOT}{\neg\,}
\newcommand{\AND}{\, \land \,}
\newcommand{\OR}{\, \lor \,}
\newenvironment{forward}{($\implies$)}{}
\newenvironment{backward}{($\limplies$)}{}

% % Sets
\DeclareMathOperator{\supp}{supp}
\newcommand{\set}[1]{\left\{#1\right\}}
\newcommand{\st}{\,|\,}
\newcommand{\minus}{\setminus}
\newcommand{\subs}{\subseteq}
\newcommand{\ssubs}{\subsetneq}
\newcommand{\sups}{\supseteq}
\newcommand{\ssups}{\supset}
\DeclareMathOperator{\im}{Im}
\newcommand{\nothing}{\varnothing}
\DeclareMathOperator{\join}{\sqcup}
\DeclareMathOperator{\meet}{\sqcap}

% % Greek 
\newcommand{\al}{\alpha}
\newcommand{\be}{\beta}
\newcommand{\ga}{\gamma}
\newcommand{\de}{\delta}
\newcommand{\ep}{\varepsilon}
\newcommand{\ph}{\varphi}
\newcommand{\io}{\iota}
\newcommand{\ka}{\kappa}
\newcommand{\la}{\lambda}
\newcommand{\om}{\omega}
\newcommand{\si}{\sigma}

\newcommand{\Ga}{\Gamma}
\newcommand{\De}{\Delta}
\newcommand{\Th}{\Theta}
\newcommand{\La}{\Lambda}
\newcommand{\Si}{\Sigma}
\newcommand{\Om}{\Omega}

% % Mathbb
\newcommand{\A}{\mathbb{A}}
\newcommand{\C}{\mathbb{C}}
\newcommand{\F}{\mathbb{F}}
\newcommand{\G}{\mathbb{G}}
\newcommand{\M}{\mathbb{M}}
\newcommand{\N}{\mathbb{N}}
\renewcommand{\P}{\mathbb{P}}
\newcommand{\Q}{\mathbb{Q}}
\newcommand{\R}{\mathbb{R}}
\newcommand{\U}{\mathbb{U}}
\newcommand{\V}{\mathbb{V}}
\newcommand{\Z}{\mathbb{Z}}

% % Mathcal
\newcommand{\BB}{\mathcal{B}}
\newcommand{\CC}{\mathcal{C}}
\newcommand{\DD}{\mathcal{D}}
\newcommand{\EE}{\mathcal{E}}
\newcommand{\FF}{\mathcal{F}}
\newcommand{\GG}{\mathcal{G}}
\newcommand{\HH}{\mathcal{H}}
\newcommand{\II}{\mathcal{I}}
\newcommand{\JJ}{\mathcal{J}}
\newcommand{\KK}{\mathcal{K}}
\newcommand{\LL}{\mathcal{L}}
\newcommand{\MM}{\mathcal{M}}
\newcommand{\NN}{\mathcal{N}}
\newcommand{\OO}{\mathcal{O}}
\newcommand{\PP}{\mathcal{P}}
\newcommand{\QQ}{\mathcal{Q}}
\newcommand{\RR}{\mathcal{R}}
\newcommand{\TT}{\mathcal{T}}
\newcommand{\UU}{\mathcal{U}}
\newcommand{\VV}{\mathcal{V}}
\newcommand{\WW}{\mathcal{W}}
\newcommand{\XX}{\mathcal{X}}
\newcommand{\YY}{\mathcal{Y}}
\newcommand{\ZZ}{\mathcal{Z}}

% % Mathfrak
\newcommand{\f}[1]{\mathfrak{#1}}

% % Mathrsfs
\newcommand{\s}[1]{\mathscr{#1}}

% % Category Theory
\DeclareMathOperator{\obj}{Obj}
\DeclareMathOperator{\END}{End}
\DeclareMathOperator{\AUT}{Aut}
\newcommand{\CAT}{\mathbf{Cat}}
\newcommand{\SET}{\mathbf{Set}}
\newcommand{\TOP}{\mathbf{Top}}
\newcommand{\MON}{\mathbf{Mon}}
\newcommand{\GRP}{\mathbf{Grp}}
\newcommand{\AB}{\mathbf{Ab}}
\newcommand{\RING}{\mathbf{Ring}}
\newcommand{\CRING}{\mathbf{CRing}}
\newcommand{\MOD}{\mathbf{Mod}}
\newcommand{\VEC}{\mathbf{Vec}}
\newcommand{\ALG}{\mathbf{Alg}}
\newcommand{\PSH}{\mathbf{PSh}}
\newcommand{\SH}{\mathbf{Sh}}
\newcommand{\ORD}{\mathbf{Ord}}
\newcommand{\POSET}{\mathbf{PoSet}}
\newcommand{\id}[1]{\mathbbm{1}_{#1}}
\newcommand{\map}[2]{\yrightarrow[#2][2.5pt]{#1}[-1pt]}
\newcommand{\iso}[1][]{\cong_{#1}}
\newcommand{\op}{^{op}}
\newcommand{\darrow}{\downarrow}
\newcommand{\LIM}{\varprojlim}
\newcommand{\COLIM}{\varinjlim}
\DeclareMathOperator{\coker}{coker}
\newcommand{\fall}[2]{\downarrow_{#2}^{#1}}
\newcommand{\lift}[2]{\uparrow_{#1}^{#2}}

% % Algebra
\newcommand{\nsub}{\trianglelefteq}
\newcommand{\inv}{^{-1}}
\newcommand{\dvd}{\,|\,}
\DeclareMathOperator{\ev}{ev}

% % Analysis
\newcommand{\abs}[1]{\left\vert #1 \right\vert}
\newcommand{\norm}[1]{\left\Vert #1 \right\Vert}
\renewcommand{\bar}[1]{\overline{#1}}
\newcommand{\<}{\langle}
\renewcommand{\>}{\rangle}
\renewcommand{\hat}[1]{\widehat{#1}}
\renewcommand{\check}[1]{\widecheck{#1}}
\newcommand{\dsum}[2]{\sum_{#1}^{#2}}
\newcommand{\dprod}[2]{\prod_{#1}^{#2}}
\newcommand{\del}[2]{\frac{\partial#1}{\partial#2}}
\newcommand{\res}[2]{{% we make the whole thing an ordinary symbol
  \left.\kern-\nulldelimiterspace % automatically resize the bar with \right
  #1 % the function
  %\vphantom{\big|} % pretend it's a little taller at normal size
  \right|_{#2} % this is the delimiter
  }}

% % Galois
\DeclareMathOperator{\gal}{Gal}
\DeclareMathOperator{\Orb}{Orb}
\DeclareMathOperator{\Stab}{Stab}
\newcommand{\emb}[3]{\mathrm{Emb}_{#1}(#2, #3)}
\newcommand{\Char}[1]{\mathrm{Char}#1}

%% code from mathabx.sty and mathabx.dcl to get some symbols from mathabx
\DeclareFontFamily{U}{mathx}{\hyphenchar\font45}
\DeclareFontShape{U}{mathx}{m}{n}{
      <5> <6> <7> <8> <9> <10>
      <10.95> <12> <14.4> <17.28> <20.74> <24.88>
      mathx10
      }{}
\DeclareSymbolFont{mathx}{U}{mathx}{m}{n}
\DeclareFontSubstitution{U}{mathx}{m}{n}
\DeclareMathAccent{\widecheck}{0}{mathx}{"71}

% Arrows with text above and below with adjustable displacement
% (Stolen from Stackexchange)
\newcommandx{\yaHelper}[2][1=\empty]{
\ifthenelse{\equal{#1}{\empty}}
  % no offset
  { \ensuremath{ \scriptstyle{ #2 } } } 
  % with offset
  { \raisebox{ #1 }[0pt][0pt]{ \ensuremath{ \scriptstyle{ #2 } } } }  
}

\newcommandx{\yrightarrow}[4][1=\empty, 2=\empty, 4=\empty, usedefault=@]{
  \ifthenelse{\equal{#2}{\empty}}
  % there's no text below
  { \xrightarrow{ \protect{ \yaHelper[ #4 ]{ #3 } } } } 
  % there's text below
  {
    \xrightarrow[ \protect{ \yaHelper[ #2 ]{ #1 } } ]
    { \protect{ \yaHelper[ #4 ]{ #3 } } } 
  } 
}

% xcolor
\definecolor{darkgrey}{gray}{0.10}
\definecolor{lightgrey}{gray}{0.30}
\definecolor{slightgrey}{gray}{0.80}
\definecolor{softblue}{RGB}{30,100,200}

% hyperref
\hypersetup{
      colorlinks = true,
      linkcolor = {softblue},
      citecolor = {blue}
}

\newcommand{\link}[1]{\hypertarget{#1}{}}
\newcommand{\linkto}[2]{\hyperlink{#1}{#2}}

% Perpage
\MakePerPage{footnote}

% Theorems

% % custom theoremstyles
\newtheoremstyle{definitionstyle}
{5pt}% above thm
{0pt}% below thm
{}% body font
{}% space to indent
{\bf}% head font
{\vspace{1mm}}% punctuation between head and body
{\newline}% space after head
{\thmname{#1}\thmnote{\,\,--\,\,#3}}

\newtheoremstyle{exercisestyle}%
{5pt}% above thm
{0pt}% below thm
{\it}% body font
{}% space to indent
{\it}% head font
{.}% punctuation between head and body
{ }% space after head
{\thmname{#1}\thmnote{ (#3)}}

\newtheoremstyle{examplestyle}%
{5pt}% above thm
{0pt}% below thm
{\it}% body font
{}% space to indent
{\it}% head font
{.}% punctuation between head and body
{\newline}% space after head
{\thmname{#1}\thmnote{ (#3)}}

\newtheoremstyle{remarkstyle}%
{5pt}% above thm
{0pt}% below thm
{}% body font
{}% space to indent
{\it}% head font
{.}% punctuation between head and body
{ }% space after head
{\thmname{#1}\thmnote{\,\,--\,\,#3}}

% Custom Environments

% % Theorem environments

\theoremstyle{definitionstyle}
\newmdtheoremenv[
    linewidth = 2pt,
    leftmargin = 20pt,
    rightmargin = 0pt,
    linecolor = darkgrey,
    topline = false,
    bottomline = false,
    rightline = false,
    footnoteinside = true
]{dfn}{Definition}
\newmdtheoremenv[
    linewidth = 2 pt,
    leftmargin = 20pt,
    rightmargin = 0pt,
    linecolor = darkgrey,
    topline = false,
    bottomline = false,
    rightline = false,
    footnoteinside = true
]{prop}{Proposition}
\newmdtheoremenv[
    linewidth = 2 pt,
    leftmargin = 20pt,
    rightmargin = 0pt,
    linecolor = darkgrey,
    topline = false,
    bottomline = false,
    rightline = false,
    footnoteinside = true
]{cor}{Corollary}

\theoremstyle{exercisestyle}
\newmdtheoremenv[
    linewidth = 0.7 pt,
    leftmargin = 20pt,
    rightmargin = 0pt,
    linecolor = darkgrey,
    topline = false,
    bottomline = false,
    rightline = false,
    footnoteinside = true
]{ex}{Exercise}
\newmdtheoremenv[
    linewidth = 0.7 pt,
    leftmargin = 20pt,
    rightmargin = 0pt,
    linecolor = darkgrey,
    topline = false,
    bottomline = false,
    rightline = false,
    footnoteinside = true
]{lem}{Lemma}

\theoremstyle{examplestyle}
\newmdtheoremenv[
    linewidth = 0.7 pt,
    leftmargin = 20pt,
    rightmargin = 0pt,
    linecolor = darkgrey,
    topline = false,
    bottomline = false,
    rightline = false,
    footnoteinside = true
]{eg}{Example}
\newmdtheoremenv[
    linewidth = 0.7 pt,
    leftmargin = 20pt,
    rightmargin = 0pt,
    linecolor = darkgrey,
    topline = false,
    bottomline = false,
    rightline = false,
    footnoteinside = true
]{ceg}{Counter Example}

\theoremstyle{remarkstyle}
\newtheorem{rmk}{Remark}

\newenvironment{proof1}{
  \begin{proof}\renewcommand\qedsymbol{$\blacksquare$}
}{
  \end{proof}
} % Proofs ending with black qedsymbol 

% % tikzcd diagram 
\newenvironment{cd}{
    \begin{figure}[H]
    \centering
    \begin{tikzcd}
}{
    \end{tikzcd}
    \end{figure}
}

% tikzcd
% % Substituting symbols for arrows in tikz comm-diagrams.
\tikzset{
  symbol/.style={
    draw=none,
    every to/.append style={
      edge node={node [sloped, allow upside down, auto=false]{$#1$}}}
  }
}

\begin{document}

\title{Notes on Bi-Quadratic Extensions}

\author{Ken Lee}
\date{Spring 2021}
\maketitle

%\tableofcontents

\begin{prop}[Classification of Bi-Quadratic Extensions]

  Let $a, b \in K$ a field of characteristic not $2$ and 
  $b$ non-square in $K$.
  Let $K \to L$ be a splitting field of $(X^2 - a)^2 - b$.
  The roots are then $\pm \sqrt{a \pm \sqrt{b}}$ and we have the tower,
  \begin{cd}
      &
        & K(\sqrt{a + \sqrt{b}}) \ar[rd]
          & \\
    K \ar[r,"2"]
      & K(\sqrt{b}) \ar[ru] \ar[rd]
        & 
          & L \\
      &
        & K(\sqrt{a - \sqrt{b}}) \ar[ru]
          &
  \end{cd}
  Define the mysterious $c := a^2 - b$.
  Then the following are true : 
  \begin{enumerate}
    \item Let $c$ be a square in $K$. Then we have cases : 
    \begin{enumerate}
      \item Let either $2(a + \sqrt{c}), 2(a - \sqrt{c})$ square in $K$.
      (We cannot have both since $b$ non-square in $K$.)
      Then $L = K(\sqrt{b})$ and hence $\AUT_K L \iso C_2$.
      \item Let both $2(a \pm \sqrt{c})$ non-squares in $K$.
      Then $[L : K] = 4$ and $\AUT_K L \iso C_2 \times C_2$.
    \end{enumerate}
    \item Let $c$ non-square in $K$. Then we have cases : 
    \begin{enumerate}
      \item Let $bc$ square in $K$. Then $[L : K] = 4$ and $\AUT_K L \iso C_4$.
      \item Let $bc$ non-square in $K$.
      Then $[L : K] = 8$ and $\AUT_K L \iso D_{2(4)}$.
    \end{enumerate}
  \end{enumerate}
\end{prop}
\begin{proof}
  Throughout the proof,
  we use $\al, \al'$ to denote $\sqrt{a + \sqrt{b}}, \sqrt{a - \sqrt{b}}$
  respectively. 
  We first give an ansatz of what $\si \in \AUT_K L$ could do 
  by exploiting $\AUT_K L \hookrightarrow \AUT\set{\pm\al,\pm\al'}$.
  Let $\si \in \AUT_K L$ which we view as a subgroup of the permutations of 
  the roots. We have the cases : 
  \begin{itemize}
    \item $\si(\al) = \al$. Then $\si(-\al) = -\al$.
    So we have two cases : 
    \begin{itemize}
      \item $\si = \id{}$
      \item $\si = (\al' -\al')$, a reflection.
    \end{itemize}
    \item $\si(\al) = \al'$. Then $\si(-\al) = -\al'$.
    So we have two cases : 
    \begin{itemize}
      \item $\si = (\al \,\,\al') (-\al \,\, -\al')$, a diagonal reflection.
      \item $\si = (\al \,\, \al' \,\, -\al \,\, -\al')$.
    \end{itemize}
    \item $\si(\al) = -\al$. Then we have either : 
    \begin{itemize}
      \item $\si = (\al \,\, -\al)$, a reflection.
      \item $\si = (\al \,\, -\al) (\al' \,\, -\al')
      = (\al \,\, \al' \,\, -\al \,\, -\al')^2$.
    \end{itemize}
    \item $\si(\al) = -\al'$. Then either : 
    \begin{itemize}
      \item $\si = (\al \,\, -\al') (-\al \,\, \al')$, a diagonal reflection.
      \item $\si = (\al \,\, -\al' \,\, -\al \,\, \al')
      = (\al \,\, \al' \,\, -\al \,\, -\al')\inv$
    \end{itemize}
  \end{itemize}
  So we see $\AUT_K L \iso D_{2(4)}$ when its cardinality is largest,
  otherwise we obtain subgroups of $D_{2(4)}$.

  The Galois group $\AUT_K L$ depends crucially on two things : 
  \begin{enumerate}
    \item [(Q1)] Are $K(\sqrt{b}) \to K(\al)$, 
    $K(\sqrt{b}) \to K(\al')$ trivial?
    \item [(Q2)] Is $K(\al) = K(\al')$?
  \end{enumerate} 
  Considering (Q1), let $a \pm \sqrt{b}$ is a square in $K(\sqrt{b})$.
  Then we have $x, y \in K$ such that 
  \begin{align*}
    a \pm \sqrt{b} 
    &= (x + y \sqrt{b})^2 = x^2 + b y^2 + 2 xy \sqrt{b} 
    \implies a = x^2 + b\brkt{\frac{\pm 1}{2x}}^2 \\
    \implies 0 &= x^4 - a x^2 + b/4 
    \implies x^2 = \frac{a \pm \sqrt{c}}{2}
  \end{align*}
  where $c := a^2 - b$, the discriminant of the quadratic in $x^2$,
  which is not so mysterious afterall.

  \textit{(Assume $c$ square in $K$)}
  \textit{(One of $2(a \pm \sqrt{c})$ square in $K$)}
  Suppose $2(a + \sqrt{c})$ square in $K$.
  Then $K(\al) = K(\sqrt{b})$.
  Since $\al \al' = \sqrt{c} \in K$, 
  we hence have $L = K(\al,\al') = K(\sqrt{b})$.
  As for the Galois group, 
  since $[L : K] = 2$, we cannot have any 4-cycles in $\AUT_K L$.
  This leaves the identity, 
  the reflections and $(\al \,\, -\al) (\al' \,\, -\al')$.
  Any automorphism fixing $\sqrt{b}$ or $\al$ or $\al'$ 
  must be identity since $L = K(\sqrt{b}) = K(\al) = K(\al')$.
  This leaves the identity and 
  two reflections diagonal reflections.
  Finally, which reflection is in $\AUT_K L$ hinges on 
  which of $\al + \al'$ or $\al - \al'$ is in $K$.
  This is the other significance of $2(a + \sqrt{c})$ : 
  $(\al + \al')^2 = 2(a + \sqrt{c})$.
  So by assumption $\al + \al' \in K$.
  Now if $(\al \,\,-\al') (\al' \,\, -\al) \in \AUT_K L$,
  we would have $\al - \al' \in K$ and hence $\al \in K$, a contradiction.
  Thus, $\AUT_K L$ is generated by $(\al \,\,\al') (-\al \,\, -\al')$.

  The above argument can be repeated instead with assuming 
  $2(a - \sqrt{c})$ is a square in $K$.
  Then everything is the same until the last step, 
  where we find $\AUT_K L$ is generated by $(\al \,\, -\al') (-\al \,\, \al')$
  instead. 

  \textit{(Both $2(a \pm \sqrt{c})$ non-square in $K$)}
  Both $a \pm \sqrt{b}$ are not squares in $K(\sqrt{b})$,
  so $K(\sqrt{b}) \to K(\al), K(\sqrt{b}) \to K(\al')$ are both degree $2$.
  But again $\al \al' = \sqrt{c} \in K$, 
  so $L = K(\al,\al') = K(\al) = K(\al')$ and hence $[L : K] = 4$.
  Any automorphism fixing $\al$ or $\al'$ must be identity,
  so this excludes the non-diagonal reflections.
  Since $(\al \pm \al')^2 = 2(a \pm \sqrt{c})$,
  we have at least three sub-$K$-extensions $K(\sqrt{b}),K(\al \pm \al')$.
  By Galois theory, this gives at least three subgroups of $\AUT_K L$
  with index $2$.
  It thus cannot be the case that $\AUT_K L$ is generated by 
  the 4-cycles,
  and hence $\AUT_K L$ is generated by the diagonal reflections,
  isomorphic to $V_4$.

  \textit{($c$ non-square in $K$)}
  The extensions $K(\sqrt{b}) \to K(\al), K(\sqrt{b}) \to K(\al')$ must 
  be order $2$.
  The question is $(Q2)$.
  After thinking for a while, 
  a thing one might try is the following : 
  \begin{lem}
    Let $F$ be a field with non-even characteristic.
    Let $r,s \in F$ with $s$ non-square in $F$.
    Then $\sqrt{r} \in F(\sqrt{s})$ if and only if 
    $rs$ or $r$ is a square in $F$.
    \begin{proof1}
      $(\implies)$
      Let $r = (x + y\sqrt{s})^2$ with $x,y \in F$.
      We get $r = x^2 + sy^2$ and $0 = 2 xy$. 
      So either $r = x^2$ or $rs = (s y)^2$.

      $(\limplies)$
      $r = x^2 / s = (x / \sqrt{s})^2$. 
    \end{proof1}
  \end{lem}
  So we have $\al' \in K(\al) = K(\sqrt{b})(\al)$ if and only if 
  $c = \al \al'$ or $a - \sqrt{b}$ square in $K(\sqrt{b})$.
  The latter implies $(a - \sqrt{c})/2$ is a square in $K$
  and hence $\sqrt{c} \in K$, a contradiction.
  So $K(\al') = K(\al)$ if and only if $c$ is a square in $K(\sqrt{b})$.
  But this is if and only if $\sqrt{c} \in K(\sqrt{b})$,
  which by another application of the lemma,
  is equivalent to $bc$ or $c$ being a square in $K$.
  The latter is again false by assumption,
  so we have $K(\al) = K(\al')$ if and only if $bc$ is a square in $K$.

  \textit{($bc$ is a square in $K$)}
  We have $L = K(\al) = K(\al')$ and $[L : K] = 4$.
  Again, any automorphism fixing $\al$ or $\al'$ must be identity 
  so we are left with the rotations and the diagonal reflections.
  If $\AUT_K L$ is generated by the diagonal reflections,
  then $[K(\al \pm \al') : K]$ would be order two and hence 
  $\min(\al\pm\al',K) = X^2 - 2(a \pm \sqrt{c})$,
  in particular $\sqrt{c} \in K$, a contradiction. 
  So $\AUT_K L$ is generated by a 4-cycle.

  \textit{($bc$ is non-square in $K$)} Covered.
\end{proof}

%\printbibliography

\end{document}